\documentclass[12pt]{article}

\usepackage{sbc-template}

\usepackage{graphicx,url}

% \usepackage[brazil]{babel}   
\usepackage[utf8]{inputenc}  
\usepackage{float}

%Define the listing package
\usepackage{listings} %code highlighter
\usepackage{color} %use color
\definecolor{mygreen}{rgb}{0,0.6,0}
\definecolor{mygray}{rgb}{0.5,0.5,0.5}
\definecolor{mymauve}{rgb}{0.58,0,0.82}

%Customize a bit the look
\lstset{ %
backgroundcolor=\color{white}, % choose the background color; you must add \usepackage{color} or \usepackage{xcolor}
basicstyle=\footnotesize, % the size of the fonts that are used for the code
breakatwhitespace=false, % sets if automatic breaks should only happen at whitespace
breaklines=true, % sets automatic line breaking
captionpos=b, % sets the caption-position to bottom
commentstyle=\color{mygreen}, % comment style
deletekeywords={...}, % if you want to delete keywords from the given language
escapeinside={\%*}{*)}, % if you want to add LaTeX within your code
extendedchars=true, % lets you use non-ASCII characters; for 8-bits encodings only, does not work with UTF-8
frame=single, % adds a frame around the code
keepspaces=true, % keeps spaces in text, useful for keeping indentation of code (possibly needs columns=flexible)
keywordstyle=\color{blue}, % keyword style
% language=Octave, % the language of the code
morekeywords={*,...}, % if you want to add more keywords to the set
numbers=left, % where to put the line-numbers; possible values are (none, left, right)
numbersep=5pt, % how far the line-numbers are from the code
numberstyle=\tiny\color{mygray}, % the style that is used for the line-numbers
rulecolor=\color{black}, % if not set, the frame-color may be changed on line-breaks within not-black text (e.g. comments (green here))
showspaces=false, % show spaces everywhere adding particular underscores; it overrides 'showstringspaces'
showstringspaces=false, % underline spaces within strings only
showtabs=false, % show tabs within strings adding particular underscores
stepnumber=1, % the step between two line-numbers. If it's 1, each line will be numbered
stringstyle=\color{mymauve}, % string literal style
tabsize=2, % sets default tabsize to 2 spaces
title=\lstname % show the filename of files included with \lstinputlisting; also try caption instead of title
}
%END of listing package%

\definecolor{darkgray}{rgb}{.4,.4,.4}
\definecolor{purple}{rgb}{0.65, 0.12, 0.82}

%define Javascript language
\lstdefinelanguage{JavaScript}{
keywords={typeof, new, true, false, catch, function, return, null, catch, switch, var, if, in, while, do, else, case, break},
keywordstyle=\color{blue}\bfseries,
ndkeywords={class, export, boolean, throw, implements, import, this},
ndkeywordstyle=\color{darkgray}\bfseries,
identifierstyle=\color{black},
sensitive=false,
comment=[l]{//},
morecomment=[s]{/*}{*/},
commentstyle=\color{purple}\ttfamily,
stringstyle=\color{red}\ttfamily,
morestring=[b]',
morestring=[b]"
}

\lstset{
language=JavaScript,
extendedchars=true,
basicstyle=\footnotesize\ttfamily,
showstringspaces=false,
showspaces=false,
numbers=left,
numberstyle=\footnotesize,
numbersep=9pt,
tabsize=2,
breaklines=true,
showtabs=false,
captionpos=b
}

     
\sloppy

\title{T2 - Segurança e auditoria de sistemas - Criptografia}

\author{Hariel Giacomuzzi Dias, Paulo Vianna Mu}


\address{Faculdade de Informática -- Centro Universitário Ritter dos Reis
  (Uniritter)\\
  Av. Manoel Elias, 2001 -- Porto Alegre -- RS -- Brazil
}

\begin{document} 

\maketitle

\begin{abstract}
In this article we will look at concepts and characteristics related to security and systems audits. More specifically we will see encryption, cryptographic summary, digital signature, public key certificate.
\end{abstract}
     
\begin{resumo} 
  Neste artigo veremos alguns detalhes práticos da implementação da criptografia de chave pública utilizando o algoritmo RSA.
\end{resumo}

\section{Criptografia Assimétrica}
Um dos maiores problemas da criptografia simétrica é a transmissão da chave para decodificação da mensagem, uma vez que não há maneiras seguras de se entrega-lá ao destinatário. Pensando-se nisso foi proposta uma abordagem diferente, a criptografia assimétrica onde temos duas chaves, uma pública e uma privada, desta forma poderíamos trocar chaves com nosso destinatário sem se preocupar se a comunicação fora interceptada, já que uma chave pode somente codificar e a outra decodificar. Assim sendo, as duas partes da comunicação devem criar um par de chaves, uma pública que será enviada para o outro usuário e uma privada, após os dois terem trocado suas chaves públicas basta usa-las para codificar suas mensagens pois a mesma mensagem só poderá ser decodificada se utilizada a chave privada do outro usuário.

\subsection{O Algoritmo RSA}
Atualmente o algoritmo mais utilizado no processo de criptografia assimétrica é o RSA\cite{stallingscriptografia}, o algoritmo é relativamente trivial de se compreender entretanto computacionalmente intenso pois o mesmo utiliza cálculos envolvendo números primos de grande magnitude. Os passos são os seguintes:

\begin{itemize}
	\item Selecione dois números primos $p$ e $q$
    \item Calcule $n = p*q$
    \item Selecione um inteiro $e$ tal que $1 < e < \varphi(n)$ de forma que $e$ e $\varphi(n)$ sejam co-primos
    \item A chave publica será então composta por $n$ e $e$
    \item A chave privada $(d)$ será igual à $(2*\varphi(n)+1)/e$
\end{itemize}

Os processos de codificação e decodificação são então dados pelos seguintes cálculos:

\begin{itemize}
	\item Codificação: a mensagem $c$ é igual ao resto da divisão por $n$ de $m^e$
    \item Decodificação: a mensagem $m$ será dada pelo resto da divisão por $n$ de $c^d$
\end{itemize}

\section{Ferramentas utilizadas}
Para realizar este exercício, utilizaremos as seguintes ferramentas:

\begin{itemize}
    \item Node.JS 11.2.0
    \item node-rsa
\end{itemize}

\subsection{Node.JS}
Utilizaremos a linguagem \textit{javascript}, por uma questão de familiaridade e também de facilidade de uso da mesma em se tratando de programadores iniciantes. A linguagem por si só não oferece grande vantagem, por isso utilizaremos o framework Node.JS \cite{node}. Este oferece um conjunto de funções e mecanismos que auxiliam o desenvolvedor, bem como um gerenciador de pacotes automatizado.

\subsection{node-rsa}
Também será adicionada uma biblioteca \cite{node-rsa} com as principais funções da criptografia de chave publica afim de minimizar problemas de implementação e lógica no decorrer da atividade.

\section{Objetivos}
Afim de uma melhor organização da atividade o planejamento da mesma será descrito nesta secção.

\begin{itemize}
    \item Criar um par de chaves para o usuário A
    \item Criar um par de chaves para o usuário B
    \item Realizar a troca das chaves públicas dos usuários
    \item Criptografar uma mensagem simples utilizando a chave pública do usuário ao qual se destina a mensagem
    \item De-criptografar a mensagem recebida utilizando a chave privada do usuário.
\end{itemize}

\section{Criando os pares de chaves}
Primeiramente é preciso importar a biblioteca que será utilizada, para fazer isso basta utilizar o comando \textit{require} e com o auxilio da palavra reservada \textit{const} criar uma constante para armazenar a referência da biblioteca importada. 

\begin{lstlisting}[language=JavaScript]
const RSA = require('node-rsa');
\end{lstlisting}

Uma vez que importamos a biblioteca, basta criarmos uma nova instância da mesma e utilizar a rotina \textit{generateKeyPair} passando como parâmetro a quantidade de bits da chave a ser gerada. Neste exemplo, por questões de performance, utilizaremos uma chave de apenas 512 bits. 

\begin{lstlisting}[language=JavaScript]
// generate key pair for user A
const keyA = new RSA();

const keyPairA = keyA.generateKeyPair(512);

console.log(`Private Key A:\n ${keyPairA.exportKey('pkcs1-private-pem')}\n`)
console.log(`Public Key A:\n ${keyPairA.exportKey('pkcs1-public-pem')}\n`)
console.log(`Maximum message size: ${keyA.getMaxMessageSize()}\n`)
\end{lstlisting}

\section{Importando uma chave pública}
Agora que temos um par de chaves para o usuário A, podemos utilizar as mesmas rotinas para a criação dos pares de chaves do usuário B. 

\begin{lstlisting}[language=JavaScript]
// generate key pair for user B
const keyB = new RSA();

const keyPairB = keyB.generateKeyPair(512);

console.log(`Private Key B:\n ${keyPairB.exportKey('pkcs1-private-pem')}\n`)
console.log(`Public Key B:\n ${keyPairB.exportKey('pkcs1-public-pem')}\n`)
console.log(`Maximum message size: ${keyB.getMaxMessageSize()}\n`)
\end{lstlisting}

E por fim, para que os usuários A e B possam trocar mensagens sigilosas, basta que eles exportem suas chaves públicas e troquem suas chaves entre si. Para isso, utilizamos as funções \textit{importKey} & \textit{exportKey}, para importar e exportar chaves. A biblioteca possui uma DSL para auxiliar na formatação de chaves para exportação e importação, mais detalhes em \cite{node-rsa}. 

\begin{lstlisting}[language=JavaScript]
// as user B import user A public key
keyB.importKey(keyPairA.exportKey('pkcs1-public-pem'), 'pkcs1-public-pem');

// as user A import user B public key
keyA.importKey(keyPairB.exportKey('pkcs1-public-pem'), 'pkcs1-public-pem');
\end{lstlisting}

\section{Encriptação & decriptação}
Agora que os usuários trocaram as chaves basta utilizar as funções \textit{encrypt} & \textit{decrypt} para que possam cifrar e decifrar suas mensagens com seguraça. 

\begin{lstlisting}[language=JavaScript]
// as user B encripts a message
let message = keyB.encrypt('this is a secret message');
console.log(`Message encripted: ${JSON.stringify(message.toString('utf8'))}\n`)

// as user A decripts user B message
let messageDecripted = keyA.decrypt(message);
console.log(`Message decripted: ${JSON.stringify(messageDecripted.toString('utf8'))}`)
\end{lstlisting}


% \begin{figure}[h]
% \includegraphics[scale=0.4]{fig/citala_2.png}
% \includegraphics[scale=0.4]{fig/citala_1.png}
% \centering
% \end{figure}

\bibliographystyle{sbc}
\bibliography{sbc-template}

\end{document}
